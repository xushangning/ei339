\documentclass[a4paper]{article}

\usepackage[margin=1in]{geometry}
\usepackage{CJK}

\title{Report for EI339 Artificial Intelligence: Homework 2}
\author{517030910384 徐尚宁}
\date{}

\begin{document}

\begin{CJK}{UTF8}{gbsn}
    \maketitle
\end{CJK}

\section*{Q1}

We can formulate the maze problem as a state-space search problem, where each
square $(x, y)$ on the coordinate system is a state and moving between adjacent
squares has a constant action cost. Some squares on the map are unreachable, to
which neighboring squares/states can not transition to. All reachable squares
constitute our state space.

Note that in our coordinate system, the vertical axis is the $x$-axis and the
horizontal $y$-axis. Each square's coordinate is defined to be its upper left
corner's coordinate. For example, the lower right ``exit'' square has the
coordinate $(26, 39)$.

\section*{Q2}

In our formulation, there will be cycles and all action costs are constant and
positive, so we choose BFS and A* search to find the optimal path.
\begin{enumerate}
    \item BFS visits each state in a ``breath-first'' manner. Its time
    complexity is $O(b^d)$ for branching factor $b$ and the length of the
    optimal path $d$ to the maze exit.
    \item The A* search is equipped with a heuristics from the relaxed problem
    where all walls in the maze are removed. The heuristics for square $(x, y)$
    is defined to be the $L_1$ distance between the square itself and the exit,
    that is
    \[
        h(x, y) = x + y
    \]

    The time complexity for A* is $O(n\log n)$, where $n$ is the number of
    squares closer to the entrance than the exit.
\end{enumerate}

\section*{Q3}

We choose to implement the BFS algorithm because intuitively, it seems that
nearly all squares are closer to the entrance than the exit, and we want to
verify this hypothesis.

\end{document}
