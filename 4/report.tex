\documentclass[a4paper]{article}

\usepackage{CJK}
\usepackage{amsthm}

\title{Report for EI339 Artificial Intelligence: Homework 4}
\author{517030910384 徐尚宁}
\date{}

\begin{document}

\begin{CJK}{UTF8}{gbsn}
    \maketitle
\end{CJK}

\section*{Q1}

Only when the white piece is adjacent to the black at the start can A win the
game.

\begin{proof}
    When the white and the black piece are adjacent to each other. A can simply
    move its piece to eat their opponent.

    We claim that if A is about to move and the $L_1$ distance between the white
    and black piece $d \geq 2$, then A will lose the game, which basically means
    that A will lose as long as the white piece is not adjacent to the black at
    the start.

    We propose the following strategy for B that leads to B's victory when $d
    \geq 2$.
    \begin{enumerate}
        \item For $d = 2$ and it is A's turn to move, there are two cases:
        \begin{itemize}
            \item If the black and the white piece form a straight line, no
            matter how the white moves, B just moves the black piece along the
            line one grid at a time.
            \item If the black is diagonally adjacent to the white, then B just
            imitates A's last move.
        \end{itemize}
        \item For $d > 2$, and it is A's turn to move, there are two cases:
        \begin{itemize}
            \item If the white moves away from the black, then B chooses a
            direction in which moving two grids will decrease $d$ by 2.
            \item If the white moves towards the black, and after the move, it
            is B's turn,
            \begin{itemize}
                \item If $d = 2$, either the black can simply eat the white when
                they are on a straight line, or when they diagonally adjacent to
                each other, the black piece moves two grids in a direction such
                that before and after the move, $d$ doesn't change.
                \item If $d = 3$, there must exist a direction in which the
                black moves one grid and decreases $d$ by one.
            \end{itemize}
        \end{itemize}
    \end{enumerate}

    The full proof follows below.

    When $d = 2$, there are two states for the chessboard: either the two pieces
    form a straight line, or the white is adjacent to the black diagonally. A
    can't move its piece towards the black piece (that is, choose a move that
    decreases the $L_1$ distance), because doing so in either case will allow
    the black piece to directly eat the white. Suppose that the two pieces are
    on a chessboard without boundary. For the white piece to survive, A will
    move it away from the black. We give a strategy for B on how to move with
    regard to each of the two cases.

    In the former case, the white and black form a straight line. No matter how
    A moves, B just needs to move its piece along the straight line for one step
    towards the white. If A's move was along the line, B's move will keep the
    two pieces in the same relative position. Or if A moved its piece
    perpendicular to the line, B's move will make the black piece diagonally
    adjacent to the white, which transforms this case into the latter case.

    In the latter case, B needs to imitate A's move to maintain their relative
    position. If the chessboard has no boundary, the strategy described above
    essentially leads to a draw, but the chessboard size is finite. The black
    piece will at some point encounter the boundary and, if the black and white
    are on a straight line, the black are forced to move sideways, leading to
    the former case.

    The game concludes when the white piece has no way to retreat.

    If B wins the game for $2 \leq d \leq k$, we want to prove that B still wins
    for $d \leq k + 1$. For $d = k + 1$, there are two possibilities for A's
    move: A can move the white piece towards or away from the black piece.
    \begin{itemize}
        \item If the white piece moves away from the black piece, then after A's
        move, $d = k + 2 \geq 4$ because $k \geq 2$. In this case, there must
        exist a direction in which the black piece can move 2 grids to decrease
        $d$ by 2 to $k$. Consider the rectangle formed by the white and black
        piece as its two opposite corners. Such rectangle must have one side
        whose length is greater than 2 because the rectangle's perimeter is
        equal to $d = k$. B moves its piece in such direction and then it is A's
        turn and $d = k$, so B wins.
        \item If the white piece moves towards the black piece, then $d = k$
        after A's move. If $k \geq 4$, B can still apply the strategy above
        because after such move, $d = k - 2 \geq 2$ and B wins.

        If, after A's move, $k < 4$, we have to separately discuss the cases for
        $k = 2$ or 3. If $k = 3$, B just moves its piece one grid towards the
        white piece so that at A's turn $d = 2$ and B wins. If $k = 2$, B will
        make a decision based on the state of the chessboard. If the black and
        white forms a straight line, the black piece can directly eats the
        white. If the black is adjacent to the white diagonally, there must
        exist a direction in which the white piece can move two grids such that
        before and after the move the $L_1$ distance $d$ doesn't change. Because
        after B's move, $d = 2$ and now it is A's turn, so B wins.

        There is one case where such move is impossible, that is when the white
        piece is at the corner, the black piece is diagonally adjacent to it and
        it is B's turn. We argue that this scenario can't happen under the
        assumption that A and B are rational. Because now it is B's turn, let
        this turn be the $i$th turn and consider what is A's move in turn $i -
        1$. Because the white piece can only move one grid at a time, the black
        would have been adjacent to the white at turn $i - 1$. A should have
        already won at turn $i - 1$ because the white can eat the black, and
        since B is rational, they would not have moved the black piece next to
        the white.
    \end{itemize}
    With the complete strategy for B outlined above, we conclude that A will
    only win when $d = 1$.
\end{proof}

\section*{Q2}

As it can be seen from the proof above, if A is doomed to lose, the white piece
will try to retreat to a corner such that
\begin{enumerate}
    \item the white piece can moves there faster than the black piece
    \item it has the maximum distance to the white piece among all corners that
    satisfy 1.
\end{enumerate}

Then, the white piece will move between the corner and its adjacent grid as not
to voluntarily come closer to the black.

Since essentially both the black and the white are moving to a corner, the
maximum number of rounds are bounded from above by two sides of the chessboard
i.e. $2n$ each for the black and the white, so $K \leq 4n$.

\end{document}
