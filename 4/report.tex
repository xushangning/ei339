\documentclass[a4paper]{article}

\usepackage[margin=1in]{geometry}
\usepackage{CJK}
\usepackage{amsthm}

\title{Report for EI339 Artificial Intelligence: Homework 4}
\author{517030910384 徐尚宁}
\date{}

\begin{document}

\begin{CJK}{UTF8}{gbsn}
    \maketitle
\end{CJK}

\section*{Q1}

Only when the white piece is adjacent to the black at the start can A win the
game.

\begin{proof}
    When the white and the black piece are adjacent to each other. A can simply
    move its piece to eat their opponent.

    Otherwise, A is doomed to lose. We prove this claim with induction on the
    the $L_1$ distance $d \geq 2$ between the white and black piece.

    When $d = 2$, there are two states for the chessboard: either the white is
    adjacent to the black diagonally, or they form a straight line. A can't move
    its piece towards the black piece (that is, choose a move that decreases the
    $L_1$ distance), because doing so in either case will allow the black piece
    to directly eat the white. Suppose that the two pieces are on a chessboard
    without boundary. For the white piece to survive, A will move it away from
    the black. We give a strategy for B on how to move with regard to each of
    the two cases.

    In the latter case, the white and black form a straight line. No matter how
    A moves, B just needs to move its piece along the straight line for one step
    towards the white. If A's move was along the line, B's move will keep the
    two pieces in the same relative position. Or if A moved its piece
    perpendicular to the line, B's move will make the black piece diagonally
    adjacent to the white, identical to the former case.

    In the former case, B needs to imitate A's move to maintain their relative
    position. If the chessboard has no boundary, the strategy described above
    essentially leads to a draw, but the chessboard size is finite. The black
    piece will at some time encounter the boundary and, if the black and white
    are on a straight line, the black are forced to move sideways, leading to
    the former case.

    The game concludes when the white piece has no way to retreat.

    If B wins the game for $d = k$, then for $d = k + 1$, if A moves its piece
    towards the black, or B just moves its piece in any direction for 2 grids.
    Both action will decrease $d$ by one, and then B just follows the strategy
    for $d = k$.

    Such a move is always possible for B at the 2nd round, because for $d > 2$,
    the rectangle formed with the black and white piece as its corners always
    has one side whose length is greater than 2. The black piece just needs to
    move in that direction.
\end{proof}

\section*{Q2}

As it can be seen from the proof above, if A is doomed to lose, the white piece
will try to retreat to a corner such that
\begin{enumerate}
    \item the white piece can moves there faster than the black piece
    \item it has the maximum distance to the white piece among all corners that
    satisfy 1.
\end{enumerate}

Then, the white piece will move between the corner and its adjacent grid as not
to voluntarily come closer to the black.

Since essentially both the black and the white are moving to a corner, the
maximum number of rounds are bounded from above by two sides of the chessboard
i.e. $2n$ each for the black and the white, so $K \leq 4n$.

\end{document}
